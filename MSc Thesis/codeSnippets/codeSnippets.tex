
\subsection{Writing Motor Angles to Arduino}
\begin{lstlisting}
        // Sends the appropriate signals to the Arduino to set the motor angles
        public void setMotorAngles()
        {
            // Already eliminated extreme angles in the Experiment class and no NaN values should have been added
            byte[] instructionBuffer = new byte[2];
            int oldAngle, newAngle, toWrite;
            while (anglesNotEqual())
            {
                for (int i = 0; i < 4; i++)
                {
                    instructionBuffer[0] = Convert.ToByte(i + 1);
                    oldAngle = currentMotorAngles[i];
                    newAngle = targetMotorAngles[i];

                    if (oldAngle < newAngle) oldAngle++;
                    else if (oldAngle > newAngle) oldAngle--;
                    else continue;

                    toWrite = oldAngle;

                    if (usingOffset && i == 0) toWrite += m1Offset;
                    if (usingOffset && i == 2) toWrite += m3Offset;

                    instructionBuffer[1] = Convert.ToByte(toWrite);

                    lock(arduinoLock) // So other threads don't simultaneously write to Arduino and cause incorrect serial transmission
                    {
                        currentPort.Write(instructionBuffer, 0, 2);
                    }

                    Thread.Sleep(setMotorDelay);
                    updateCurrentMotorAngles(i, oldAngle);
                }
            }
        }
\end{lstlisting}



\pagebreak
\subsection{Receiving Instructions on the Arduino}
\lstset{language=C++}
\begin{lstlisting}

void loop() {
...
  if (sentFromPC()) {
 
    int controlCode = Serial.read();
    int instruction = Serial.read();
    if(controlCode > 0 && controlCode < 5) {
      motorInstruction(controlCode, instruction);
    }
    else if(controlCode == 7) {
      triggerInstruction(instruction);
    }
  }
  
}

bool sentFromPC() {
  if(connectedToPC && Serial.available() == 2) {
    return true;
  }
  else {
    return false;
  }
}

void motorInstruction(int selectedServo, int angle) {
  switch(selectedServo) {
      case 1:
        currentServo = myservo1;
        break;
      case 2:
        currentServo = myservo2;
        break;
      case 3:
        currentServo = myservo3;
        break;
      case 4:
        currentServo = myservo4;
        break;
      default:
        break;
    }
    currentServo.write(angle);
}

void triggerInstruction(int instruction) {
  
    if(instruction == 7) {
      respondTriggerValue();
    }
    else if(instruction == 1) {
      digitalWrite(magnetControlPin, HIGH);
    }
    else if(instruction == 0) {
      digitalWrite(magnetControlPin, LOW);
    }
}

\end{lstlisting}

\subsection{Regression Function}
\lstset{language=[Sharp]C}
\begin{lstlisting}
        private double[] NWRegression(float[] inputVectorTarget, RegressionInput inputType, float alpha = startingAlpha)
        {
            int numOutputDimensions = getNumOutputDimensions(inputType);
            double[] outputVector = new double[numOutputDimensions];
            double[] sumNumerator = new double[numOutputDimensions];
            double sumDenominator = 0;
            float[] newInputVector = getCopyVector(inputVectorTarget);
           
            if (inputType == RegressionInput.MOTORS)
            {
                convertMotors(newInputVector, true);
            }

            // loop through entire set of data
            for (int i = 0; i < storedCalibrationData.Count; i++)
            {
                float kernelInput = getKernelInput(i, inputType, alpha, newInputVector);
                float[] currentYValue = getcurrentYValue(i, inputType);
                for (int k = 0; k < currentYValue.Length; k++)
                {
                    sumNumerator[k] += currentYValue[k] * kernelFunction(kernelInput);
                }
                sumDenominator += kernelFunction(kernelInput);
            }


            for (int k = 0; k < numOutputDimensions; k++)
            {
                outputVector[k] = sumNumerator[k] / sumDenominator;
            }

            if (inputType != RegressionInput.MOTORS)
            {
                convertOutputMotors(outputVector, false);
            }

            return outputVector;
        }
\end{lstlisting}

\pagebreak
\subsection{Logic and Angle Setting in Different Threads for User Study}
\begin{lstlisting}
        // Sets the user study going, starts the thread to constantly update the motor angles
        public void startStudy(UserStudyType type)
        {
            activeStudy = new UserStudy(type, randomiseGesturing);
            experimentLive = true;
            new Task(liveExperimentThreadLoop).Start();
            runUserStudy();
            experimentLive = false;
            controller.zeroMotors();
            if(type == UserStudyType.GESTURING)
            {
                activeStudy.printRandomisedOrder();
            }
        }
        
        // Repeatedly sets the motor angles (delay is present within setMotorAngles function)
        private void liveExperimentThreadLoop()
        {
            while (experimentLive)
            {
                setMotorAngles();
            }
        }
        
        // Runs user study by informing object of base position and trigger, and fetching appropriate position
        private void runUserStudy()
        {
            bool triggerPress;
            float[] relativeTargetPosition; 
            if (!activeStudy.isInitialised())
            {
                // Check that it has been initialise before attempting to run
                Console.WriteLine("unable to initialise user study");
                return;
            }
            relativeTargetPosition = activeStudy.getRelativeTargetPosition();
            runningStudy = true;
            while(runningStudy)
            {
                triggerPress = getTrigger();
                runningStudy = activeStudy.update(basePosition, triggerPress);

                if(triggerPress)
                {
                    zeroMotorAngles();
                    controller.activateMagnet(true);
                }
                else if (runningStudy)
                {
                    controller.activateMagnet(false);
                    getMotorAnglesForTargetPoint(relativeTargetPosition);
                }
                Thread.Sleep(newTargetDelay);
            }
        }

\end{lstlisting}